\documentclass[12pt]{article}
\usepackage{mathtools}
\usepackage{venndiagram}
\title{CS394 R Homework 1\\
\large attachments and source available at https://github.com/alexskc/cs394r}
\author{Aleksander Charatonik}
\begin{document}

\maketitle

\section{}
\(U=\{1,2,3\dots 10\}, A=\{1,4,7,10\}, B=\{1,2,3,4,5\}, C=\{2,4,6,8\}\).\\
\subsection*{a}
\(\bar{A} \cap C = \{2,6,8\}, |\bar{A} \cap C| = 3\)
\subsection*{b}
\(B-\bar{C} = \{2,4\}, |B-\bar{C}| = 2\)
\subsection*{c}
\(B \cup A = \{1,2,3,4,5,7,10\}, |B \cup A| = 7\)
\subsection*{d}
\(\bar{B} \cap (A - C) = \{1\}, |\bar{B} \cap (A - C)| = 1\)
\subsection*{e}
\((A - B) \cap (B - C) = \emptyset, |(A - B) \cap (B - C)| = 0\)

\section{}
\subsection*{a}
\(|A \cup B| = |A| + |B| \)\\
No. If there's any intersection between A and B, that will be counted twice.\\
\subsection*{b}
\(|A \cup B| = |A| + |B| + |A \cap B|\)\\
No. If there's any intersection between A and B, that will be counte thrice.\\
\subsection*{c}
\(|A \cup B| = |A| + |B| - |A \cap B|\)\\
Yes. This accounts for any overlap between A and B.\\
\subsection*{d}
\(|A \cup B \cup C| = |A| + |B| + |C|\)\\
No. Once again, this doesn't account for overlap.\\
\subsection*{e}
\(|A \cup B \cup C| = |A| + |B| + |C| - |A \cap B| - |A \cap C| - |B \cap C| + |A \cap B \cap C|\)\\
Yes. The intersection between each set is erased once, which means that the intersection between all three is replaced thrice, leaving a "hole." The \(|A \cap B \cap C|\) compensates for this.\\
\subsection*{f}
\(|A \cup B \cup C| = |A| + |B| + |C| - |A \cap B| - |A \cap C| - |B \cap C|\)\\
No, as mentioned above, this leaves a "hole" in the middle.\\
\subsection*{g}
\(|A \cup B \cup C| = |A| + |B| + |C|\)\\
No, this is the same as (d)

\section{}
\subsection*{a}
\(A \cup B = A\). Either \(A = B\), or \(B = \emptyset\).
\subsection*{b}
\(B - A = B\). Either \(A \cap B = \emptyset\), or \(A = \emptyset\).
\subsection*{c}
\(A - B = B - A\). \(A = B\)
\subsection*{d}
\(A \cap B = A\). \(A = B\)
\subsection*{e}
\(A \cap B = B \cap A\). Always true
\subsection*{f}
\(\bar{A} \cap U = \emptyset\). \(A = U\)
\subsection*{g}
\(A - B = \emptyset\). \(A \subset B\)
\subsection*{h}
\(A \cap B = A - B\). \(A = \emptyset\)

\section{}
\begin{venndiagram2sets}[labelA=10, labelB=12, labelAB=8, labelNotAB=80, shade=Green!30]
\end{venndiagram2sets}
\subsection*{a}
\(|U| = 110\)
\subsection*{b}
\(|A| = 18\)
\subsection*{c}
\(|A \cap B| = 8\)

\section{}
\begin{venndiagram3sets}[labelA=female, labelB=married, labelC=young,
  labelABC=8, labelOnlyBC=14, labelOnlyAB=27, labelOnlyAC=23,
  labelOnlyB=18, labelOnlyC=44, labelOnlyA=24,
  labelNotABC=42]
\end{venndiagram3sets}
\subsection{a}
47 policyholders.
\subsection{b}
14 polictyholders.
\subsection{c}
42 policyholders.
\subsection{d}
27 policyholders.
\subsection{e}
66 policyholders.
\end{document}
